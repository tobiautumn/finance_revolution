\documentclass[a4paper,12pt]{article}

\usepackage[english]{babel}
\usepackage[utf8]{inputenc}
\usepackage{graphicx} 
\usepackage[colorlinks=true,linkcolor=black, citecolor=blue, urlcolor=blue, ]{hyperref}
\usepackage{url}
\usepackage{eurosym}
\usepackage{geometry}
\geometry{a4paper,left=30mm,right=25mm,top=25mm,bottom=25mm}
\usepackage{tabularx}
\usepackage{pdflscape}
\usepackage{rotating}
\usepackage{lipsum}

% Mathematical Environment 
\usepackage{amsmath}			% ams (American Mathematical Society) setup for everything related to equations
\usepackage{amsthm}				% ams setup for theorems
\usepackage{amssymb}			% ams setup for symbols
\usepackage{mathtools}
\usepackage{booktabs}           % Needed to display Latex tables exported from Python
\usepackage{rotating}           % <-- HERE

\usepackage{listings}
\usepackage{lettrine}
\usepackage{listings}
\usepackage{color}
\usepackage{setspace} 
\usepackage{framed} 
\usepackage{tikz}
\usepackage{pgfplots}


	
\usepackage{float}
\usepackage[round]{natbib}
\bibliographystyle{ecca} 

\newcommand*{\quelle}{% 
  \footnotesize Source: 
} 

\usepackage[toc,page]{appendix}

\usepackage{etoolbox}
\appto\appendix{\addtocontents{toc}{\protect\setcounter{tocdepth}{1}}}

\definecolor{dunkelgruen}{rgb}{0,0.4,0}

\title{Why some penguins hatch stones and further fun facts about the greatest animal of all times}
\author{Tobias Herbst\thanks{tobias.herbst@uni-bonn.de} \hspace{1cm}  Martin Waibel\thanks{martin.waibel@phdstudent.hhs.se} }
\date{\today}

\begin{document}

\maketitle

\begin{abstract}
\lipsum[1-1]




\end{abstract}

\pagenumbering{arabic}
\clearpage
\section{Motivation \& Research Idea}
\label{motivation}

test

\section{Data \& Empirical Strategy}
\label{data_strategy}

\subsection{Data}
\label{data}



\subsection{Empirical Strategy}
\label{emp_strategy}

\textbf{Ideas / usefule things}
\begin{enumerate}
	\item \textbf{Treated and untreated assets:} (Most) illiquid assets are associated with a large RSF factor $\rightarrow$ Most costly for banks. This is for example the case for loans to FIs with a residual maturity of 12 months or more $\rightarrow$ This implies a "negative liquidity premium" on long-term interbank loans. We could use this for identification on the asset level.

	\item \textbf{Heterogeneous factors across asset classes:} NSFR treats liabilities, equity instruments and assets separately. Off-balance sheet exposures generally receive an RSF factor of 5 \%. Specific factors can be determined at the national level 

\end{enumerate}





\section{Questions}
\label{questions}


\clearpage

\section{Appendix}
\subsection{Net Stable Funding Ratio}

\begin{enumerate}
	\item BIS (2018)
		\begin{itemize}
			\item Failure of banks to adequately manage their liquidity risk during the financial crisis led to the implementation of the LCR and NSFR.

			\item LCR targets banks' short-term resilience

			\item NSFR aims to promote resilience over a longer time horizon by creating incentives for banks to fund their activities with \textbf{more stable sources of funding} on an ongoing basis.

			\item \textbf{Theory:}
			\begin{itemize}
				\item Banks have incentives to limit excessive reliance of unstable (often illiquid) funding of core assets. 

				\item Too high reliance on cheap and abundant short-term wholesale funding in good times

				\item Stable funding sources guarantee that banks do not experience a significant increase in distress probabilities when hit by a funding shock.
			\end{itemize}

			\item Technical expression:

			\begin{equation}
				NSFR = \frac{ASF}{RSF} \geq 100 \%
			\end{equation}

			\begin{itemize}
				\item \textbf{ASF:}

					\begin{itemize}
						\item Total ASF is the portion of its capital and liabilities that will remain with the institution for more than one year. 

						\item Supervisors will assign one of 5 ASF factors to the carrying value of each funding element (100 \%: Funding is expected to be fully available in one year; 0 \%: Most unreliable funding). Further factors are: 95 \%  90 \% and 50 \%

						\item ASF is based on the sum of the ASF amounts in each category of liabilities.
					\end{itemize}

				\item \textbf{RSF:}
					\begin{itemize}
							\item Total amount of stable funding that is required to be held given the bank's liquidity characteristics and residual maturities on its assets and the contingent liquidity risk arising from its off-balance sheet exposures

							\item For each item the RFS amount is determined by assigning an factor to the carrying value of the exposure

							\item RFS factor = 100 \% Assets or exposures need to be entirely financed by stable funding because it is illiquid (e.g. loans to FIs with residual maturity of 12 months or more). An RFS factor of $0 \%$ applies to fully liquid and unencumbered assets. The other RSF factors are 85 \%, 65 \%, 50 \%, 15 \%, 10 \% and 5\% 
						\end{itemize}
			\end{itemize}
		\end{itemize}

		\item \textbf{Timeline:}
		\begin{itemize}
			\item NSFR became a minimum standard to all internationally active banks in a consolidated basis on 1 January 2018

			\item Banks must meet the NSFR on an ongoing basis and report on a quarterly basis
		\end{itemize}

	\item \cite{King2013}

	\begin{itemize}
		\item Banks below the ratio need to increase stable sources of funding and reduce assets requiring funding.

		\item The most cost-effective strategies to meet the NSFR are to increase holdings of higher-rated securities and to extend the maturity of wholesale funding. 

		\item Universal banks with diversified funding sources and high trading assets are penalized most by the NSFR

		\item The implementation of the NSFR will lead to a reduction in the net interest rate margin (NIM) for banks.

		\item Overall intention:
			\begin{itemize}
				\item Encourage banks to hold more HQLA assets and increase funding from stable sources such as deposits, longer maturity debt and equity
			\end{itemize} 
	\end{itemize}

\end{enumerate}

\newpage
\bibliography{literature}











\end{document}