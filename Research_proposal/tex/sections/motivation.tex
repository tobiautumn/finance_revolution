After the great financial crisis, regulators across the globe have tightened financial regulation, especially banking regulation. While higher capital requirements and buffers are maybe the most prominent changes, Basel III also includes advances with respect to liquidity regulation. This comes at no surprise, given that many bank failures in the past have actually been triggered rather by liquidity than solvency shortfalls (hat Christian mal im Kurs gesagt :D) and that liquidity has long been a concern also in academic research (Diamond-Dybvig etc.).\\

Two new regulations stand out: the liquidity coverage ratio (LCR) and the net stable funding ratio (NSFR). Both, generally speaking, require banks to adjust their portfolios such that they can withstand a sudden outflow of liabilities because they hold enough high quality liquid assets (HQLA) that could be liquidated to generate the cash needed to finance those outflows.\\

As with every banking regulation, the first question that probably arises is how banks react to the new regulation, i.e. how they reshuffle their assets and liabilities. In the realm of liquidity regulation, the implications might, however, be broader since e.g. not only banks are interested in HQLA. I\textbf{n this project, we therefore ask: What are the (potentially unintended) second round effects of liquidity regulation?} \\

The LCR has by now been introduced in many countries and would hence offer a natural starting point for our research. [was jetzt kommt, habe ich mir beim Schreiben als justification ausgedacht] However, the introduction of the LCR coincides with the introduction of a lot of other new regulation and hence, isolating the effect of the liquidity regulation appears challenging. \textbf{We therefore focus on the implementation of the NSFR.}\\ 