%\documentclass[handout]{beamer} % Activate only if handout slides of presentation desired.
\documentclass{beamer}

\usetheme{CambridgeUS}
\newcommand{\heart}{\ensuremath\heartsuit}

% Packages
\usepackage[utf8]{inputenc}
\usepackage[T1]{fontenc}
\usepackage{amsmath}
\usepackage{graphicx}
\usepackage{eurosym}
\usepackage{hyperref}
\usepackage{amsmath}
\usepackage{graphicx}
\usepackage[round]{natbib}
\bibliographystyle{ecca} 

\setlength{\unitlength}{1.2cm}
\setcounter{tocdepth}{2} 
\setbeamertemplate{footline}
{
	\leavevmode%
	\hbox{%
		\begin{beamercolorbox}[wd=0.4\paperwidth,ht=2.25ex,dp=1ex,center]{author in head/foot}%
			\usebeamerfont{author in head/foot}\today
		\end{beamercolorbox}
		
		\begin{beamercolorbox}[wd=.3\paperwidth,ht=2.25ex,dp=1ex,center]{author in head/foot}%
			\usebeamerfont{title in head/foot} Martin Waibel
		\end{beamercolorbox}
		%
		\begin{beamercolorbox}[wd=.3\paperwidth,ht=2.25ex,dp=1ex,center]{title in head/foot}%
			\usebeamerfont{title in head/foot}
			\insertframenumber{} / \inserttotalframenumber\hspace*{1ex}
		\end{beamercolorbox}}%


	\vskip0pt%
}

\setbeamertemplate{navigation symbols}{}
%\beamertemplatesolidbackgroundcolor{black!5}
%\setbeamercovered{transparent}




%-------------------------------------------------------------




%-------------------------------------------------------------

\begin{document}
	
\title{Asset-side regulation}
\author{}
\date{\today}

 \renewcommand*\inserttotalframenumber{1}

\begin{frame}
\maketitle
\end{frame}


\begin{frame}
\frametitle{Roberts et al. (2018)}
\begin{itemize}
	\item \textbf{LCR:} Requires banks with total assets $\geq 50$bn to hold enough HQLA that can easily and quickly be converted into cash within 30 days during a time of financial distress
		
		\begin{itemize}
			\item Within their HQLA portfolio, banks must hold a minimum amount of most liquid assets (Level 1 assets) and also abide to Level 2A and 2B asset caps.
		\end{itemize}

	\item Banks need to only meet their liquidity requirements \textbf{in} their HQLA portfolio and could hence change the liquidity in their {non} HQLA portfolio according to their liquidity preferences.


	


\end{itemize}
\end{frame}

\begin{frame}
\frametitle{Roberts et al. (2018)}
\begin{itemize}
	

	\item Considering individual assets within each HQLA level, the authors find a greater preference for reserves within Level 1 assets. This likely reflects the greater saftey and convenience of reserves as compared to US Treasuries and other Level 1 assets.

	\item Decreased holding of loans by LCR banks after the introduction of the regulation (2013).

	\item \textbf{Empirical Measures:} Bai et al. (2018) and Berger and Bouwmann (2009) 
	

	


\end{itemize}
\end{frame}


\begin{frame}
\frametitle{Roberts et al. (2018)}
\begin{itemize}
	\item \textbf{Definitions:} The US LCR rule defines HQLA assets as non-financial assets with a low risk profile, with a large market, without sharp historical price declines, and readily valued and converted to cash in times of stress. The asset must be Liquid and readily marketable implying that it has at least two market makers, many non-market makers, readily observable prices, and high trading volumes. 
	


\end{itemize}
\end{frame}

\begin{frame}
\frametitle{BIS Definition
}
\begin{itemize}
	\item \textbf{Level 1 Assets:} HQLA are comprised of Level 1 and Level 2 assets. Level 1 assets generally include cash, central bank reserves, and certain marketable securities backed by sovereigns and central banks, among others. These assets are typically of the highest quality and the most liquid, and there is no limit on the extent to which a bank can hold these assets to meet the LCR. Level 2 assets are comprised of Level 2A and Level 2B assets. Level 2A assets include, for example, certain government securities, covered bonds and corporate debt securities. Level 2B assets include lower rated corporate bonds, residential mortgage backed securities and equities that meet certain conditions. Level 2 assets may not in aggregate account for more than 40\% of a bank’s stock of HQLA. Level 2B assets may not account for more than 15\% of a bank’s total stock of HQLA.

\end{itemize}
\end{frame}







\end{document}